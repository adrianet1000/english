\section{Lesson 8 ``Entiende la a de cabernicola''}

Hi!. how's it going?.\\
I'm Andy and I'm here to help you with your English.

dog.\\
This is a dog.\\
Esto es un perro.

Cat.\\
gato.

Un gato.\\
a cat.

Esto es un gato.\\
This is a cat.

Casa.\\
House.

una casa.\\
a house.

esto es una casa.\\
This is a house.

coche.\\
car.

esto es un coche.\\
this is a car.

mesa.\\
table.

una mesa.\\
a table.

\textcolor{red}{En todos los casos se pronuncio esta palabra con el
sonido de cabernicola.}

\textcolor{red}{En castellano distinguimos entre un y una, pero en ingles
solo se usa el sonico de cabernicola ``a''}

un perro.\\
a dog.

\textcolor{red}{Pronunciando super claramente se puede escuchar ``ei''}

si todas las palabras se pronuncian muy lentamente y por separado, es muy
común escuchar la pronunciación ``ei''.

cuando se usa la pronunciacion ``ei'' en la mayoria del mundo angloparlante
se usa como algo fuerte o enfacis en la palabra ``un'' o ``una''.

He's not the teacher.\\
He's a teacher.

o quiza hablando de manera dramatica.

revelando la respuesta jugando con los niños, por ejemplo.

El es unnnn ¡profesor!.

He is aaaaaa ¡teacher!.

En el mundo angloparlante la pronunciación ``ei'' es algo especial, es decir
que si no lo decimos al hablar cada palabra por separado y luego esta letra
lo pronunciamos con el sonido largo suena muy raro, suena muy infantil.

No obstante en Norteamerica no solo usan la pronunciación ``ei'' hablando
super claramente o por separado jugando con los niños, sino tambien cuando
hablan lentamente, se escucha mas alla que en otras partes, pero igualmente
alla se escucha mas el sonido de cabernicola cuando hablan de manera fluida.

incluso cuando al inglesito habla lentamente usa el sonido de cabernicola
cuando expresa la palabra un o una en ingles.

Profesor/profesora/maestro/maestra.\\
teacher.

articulo indefinido o articulo ``a''.\\

articulo indefinido o articulo indeterminado.

De manera natural, es decir diciendo el articulo indefinido con una
pronunciacion débil con el sonido de cabernicola. di.

un perro.\\
a dog.

lentamente como lo diria un estadounidense.

This is a table.

En norteamerica al hablar lentamente, se puede escuchar el sonido largo
sin tener que decirlo de manera especial el articulo indeterminado,
cosa que no pasa con el inglesito que tampoco usa
el sonido largo aun hablando lentamente.

por esa razon es que se suele enseñar en estados unidos esta pronunciación
en un curso de ingles inicial.

de manera enfatica, es decir haciendo enfacis di.

este es un gato.\\
this is a cat.

de manera natural di.

El es profesor.\\
He's a teacher.

Con las profesiones en ingles se utiliza el articulo indeterminado.

otra de las justificaciones para utilizar las formas largas es hablando
con una persona que no escucha bien, como lo pueden ser las personas mayores
de edad y que se les habla de manera fuerte, separando las palabras y con
mucha claridad.

He is a teacher.

bye!










