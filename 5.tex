\section{Lesson 5 ``You are y por qué usar las formas largas.''}

Hi! how's it going?.\\

I'm Andy and I'm here to help you with your english.\\
Soy Andy y estoy aquí para ayudarte con tu ingles.

You are a good friend.\\
Forma larga de ``you're''.

You are in Italy.\\
Usted esta en italia.

You are my best friend.\\
Eres mi mejor amigo.

You are eating.\\
Estas comiendo.

Italy.\\
\textcolor{red}{Acentuando la primera sílaba.}

You are from Italy.\\
Tu eres de italia.

You're from Italy.\\
Pronunciando ``You're'' con el sonido de cabernicola.

from con el sonido débil (es decir con el sonido de cavernicola).

from con el sonido fuerte se parece mas a la o del español.

Hay algunas justificaciones para usar las formas largas.

En el contexto de que una persona pudo venir a tu fiesta y era muy dificil
que venga, podemos decir.

You are here!

Acentuando el verbo ``are'' pero con el sonido de cabernicola.

En otro contexto, por ejemplo en un mapa de una ciudad, es muy común
ver que escrito ``You are here'', cuando lees esto en un mapa en ingles,
lo normal es ver la forma larga.

La justificación aquí es que es escritura formal. en las publicaciones, mapas
podemos ver este tipo de escritura usando las formas largas.

La escritura formal es algo que consumimos con bastante frecuencia.

Pero que solemos producir con muy poca frecuencia.

Si eres periodista, usas estas formas largas con bastante frecuencia,
pero la mayoria de la gente no usa estas formas largas a menudo.

Si yo estuviera leyendo un mapa y quiero decir ``Yo estoy aqui''.\\
Lo diria con la forma acortada porque no estas escribiendo, estas hablando
por lo tando dirias.

I'm here.

Pero estas apuntando a otra parte con tu dedo, entonces te corrijo.

No, You're here.

y me contestas.

No, I'm here.

obviamente ya estoy harto de esto y lo digo de manera fuerte, muy
enfaticamente.

You are here.

Ahora tenemos nuestra justificación para una forma larga, hablar fuertemente
ya harto, en este caso acentue todas las palabras porque estaba arto, pero
ya habia usado las formas acortadas varias veces.

Otra justificación diciendo de manera natural diciento ``No, Ella es de italia''
de manera natural.

No, she's from Italy.

Hemos prácticado el presente continuo, tambien llamado el presente
progresivo diciendo:

You are eating.\\
Estas comiendo.

You are learning.\\
Estas aprendiendo.

















