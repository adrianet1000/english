\section{Lesson 141 CONDICIONALES NEGATIVAS}

Hi! how's it going?, welcome to lesson 141.\\
In the previous lesson you've learned that with
almost all verb the conditional form can be made
with the work would plus the infinitive.

En la lección previa aprendiste que con casi
todos los verbos se podia armar la forma condicional con
la palabra would más el infinitivo.

\begin{figure}[H]
\centering
\includegraphics{109.png}
\end{figure}

And you've seen that in English it's really easy.\\
Y has visto que en ingles es muy facil.

Spanish has all of these endings.\\
El español tiene todas estas terminaciones.

and we've got the word would.\\
y nosotros tenemos la palabra would.

I \textcolor{yellow}{would} know.\\
She \textcolor{yellow}{would} go.\\
He \textcolor{yellow}{would} tell me.\\
They \textcolor{yellow}{would} play all night.\\
We \textcolor{yellow}{would} prefer chicken.\\
You \textcolor{yellow}{would} win.

but we didn't do any negative sentences.\\
Pero no hicimos ninguna oracion negativa.

That's what we're going to do in this lesson.\\
Eso es lo que vamos a hacer en esta lección.

this is something really easy to do.\\
Esto es algo realmente fácil de hacer.

If I show you a sentence.\\
Si te muestro una oración.

I would prefer to play football.\\
Preferiría jugar al futbol.

first, What does it mean?.\\
Primero, que quiere decir.

Yo preferiría jugar al futbol.

Y en estados unidos se refieren al
futbol con los cascos si dices futbol.

What word can we add to make this sentence negative?.\\
¿Qué palabra podemos añadir para negar esta oración?.

I would not prefer to play football.\\
Yo no preferiría jugar al futbol.

easy.\\
facil.

We've already done negative sentences like this.\\
Ya hemos hecho oraciones negativas asi.

I do not want to play football.\\
No quiero jugar al futbol.

I did not want to play football.\\
No quise jugar al futbol.

have a go.\\
Pruebalo tu.

I'll give you positive sentences.\\
Yo te dare oraciones afirmativas.

First, tell me what they mean.\\
Primero, dime que quieren decir.

then make them into negative sentences.\\
Luego pasalas a oraciones negativas.

how?, easy, add the word not.\\
¿Cómo?, fácil, añade la palabra not.

Number one.\\

I would wait.\\
Yo esperaria.

I would not wait.\\
Yo no esperatia.

wait means "esperar".\\
"Wait" quiere decir esperar.

Make it negative.\\
Haslo negativo.

Number 2.\\
It would be better.\\
Eso sería mejor.

It would not be better.\\
Eso no sería mejor.

What does it mean?.\\
¿Qué quiere decir?.

Better means "mejor".\\
"Better" quiere decir mejor.

Make it negative.\\
Haslo negativo.

easy.\\
facil.

It would be better to wait.\\
Sería mejor esperar.

What does it mean?.\\
¿Qué quiere decir?.

Make it negative.\\
Haslo negativo.

It would not be better to wait.\\
No sería mejor esperar.

And number 4.\\
That would be a mistake.\\
Eso sería un error.

What does it mean?.\\
¿Qué quiere decir?.

mistake means "error".\\
"mistake" significa error.

"error" tambien existe en ingles.\\
Pero es mucho menos común, suena más oficial,
más formal. Por eso es mucho menos común.

That would be an error.\\
error con la segunda silaba pronunciada
con el sonido de cabernicola.

mistake es el rey de los errores.\\

\begin{figure}[H]
\centering
\includegraphics{110.png}
\end{figure}

make it negative.\\
Haslo negativo.

That would not be a mistake.\\
Eso no sería un error.

very good, but this is not the way we normally
do negatives with would.\\
Muy bien, pero no es asi como normalmente hacemos
los negativos con would.

It sounds strong or formal.\\
Suena fuerte o formal.

or when we speak with pauses.\\
o cuando hablamos con pausas.

Why?.\\
¿Por qué?.

Because it's a long form.\\
Porque es una forma larga.

And in English the short form are the normal ones
when they exist.\\
Y en ingles las formas acortadas son las normales
cuando existen.

Sounds strange without a special reason.\\
suena raro sin una razon especial.

How would I normally say "no hablo español" in English?.\\
¿Cómo normalmente diría yo "no hablo español" en ingles?.

With a short form.\\
Con la forma acortada.

I don't speak Spanish.\\
No hablo español.

Si hablo con las formas largas. Sueno muy raro.\\
maybe very strongly.\\
Quizá con mucha fuerza.

or very very clearly.\\
o muy claramente.

but that's not normal.\\
Pero eso no es lo normal.

The same with did not we normally say didn't.\\
Lo mismo con "did not" nosotros normalmente decimos "didn't".

And the same with "would not".\\
Y lo mismo con "with not".

We normally say wouldn't.\\
Normalmente decimos "wouldn't".

In example one we said "I would not wait".\\
En el ejemplo uno dijimos "I would not wait".

Fine, if I say it with special emphasis.\\
Bien, si lo digo con un énfasis especial.

But it's not a normal way.\\
Pero no es la manera normal.

How would we normally say this?.\\
¿Cómo normalmente diriamos esto?.

With the short form wouldn't.\\
Con l aforma acortada "wouldn't".

I wouldn't wait.\\
Yo esperaría.

repeat.\\
I wouldn't wait.

In the second example you said "it would not be better".\\
En el segundo ejemplo dijiste "it would not be better".

Say it in a more natural way.\\
Dilo de manera más natural.

It wouldn't be better.\\
Eso no sería mejor.

Repeat.\\
It wouldn't be better.

That would be much better.\\
Eso sería mucho mejor.

In the third example you said "It would not be better to wait".\\
En el tercer ejemplo dijiste "It would not be better to wait".

No sería mejor esperar.

Say it in a more natural way.\\
Dilo de manera más natural.

It wouldn't be better to wait.\\
No sería mejor esperar.

In the fourth examplo you said "That would not be a mistake".\\
En el cuarto ejemplo dijiste "That would not be a mistake".

Say it in a more natural way.\\
Dilo de manera mas natural.

That wouldn't be a mistake.\\
Eso no sería un error.

So you'll hear "wouldn't" a lot more than "would not".\\
Entonces, escucharas "wouldn't" mucho mas que "would not".

but if you read newspapers or other formal publications
you'll read "would not" more in the articles.

Pero si lees periodicos o otras publicaciones formales leeras
"would not" mas en estos articulos.

Excellent! before we finish I want to mention one thing regarding
pronunciation.

¡Excelente! antes de terminar quiero mensionar una cosa relacionada
con la pronunciación.

It's incredibly common both in Britain and America to not pronounce the T
at the end of the wouldn't.\\
Es increiblemente común tanto en Gran Bretaña como en Estados Unidos no
pronunciar la T al final de wouldn't.

I would normally say wouldn't.\\
Yo normalmente digo "wouldn't".

I trap the air.\\
atrapo el aire.

En la posición de la T pero atrapo el aire.\\

So that you get used to it.\\
Para que te acostumbres.

We'll do a few examples like that.\\
Haremos unos ejemplos así.

What do these phrases mean?.\\
¿Qué quieren decir estas frases?.

I wouldn't do that.\\
Yo no haría eso.

They wouldn't know.\\
Ellos no sabrían.

He wouldn't tell you.\\
El no te lo diría.\\
El no te contaría.

I wouldn't have a chue.\\
Yo no tendría ni idea.

así que escucharas "wouldn't" sin escuchar la t
al final y atrapando el aire.

Excellent and the useful phrase for today is just "I wouldn't".\\
Excelente y la frase útil de hoy es simplemente "I wouldn't".

It's advice.\\
Es un consejo.

En ingles si queremos decir "no lo haría yo", o cualquier condicional
negativa "no lo diría yo", "no lo compraría yo", etc, podemos simplemente
decir "I wouldn't".

Y depende del enfasis, con mas enfasis, si se quiere, se pronuncia fuerte
el pronombre "I", pero normalmente se pronuncia de manera debil este pronombre
con el sonido de cabernicola.

I'll say a few phrases things I'm going todo.\\
Yo dire unas frases cosas que voy a hacer.

Just answer.\\
Simplemente contesta.

To you it seems a bad idea.\\
A ti te parece una mala idea.

Answer "I wouldn't".\\
responde "I wouldn't"

I'm going to talk to Mariana.\\
Voy a hablar con Mariana.

I wouldn't.

I'm going to buy this car.\\
Voy a comprar este coche.

I'm going to go to that restaurant that Inoa recommended.\\
Voy a ir a ese restaurant que recomendo Inoa.

I wouldn't.

There is another short negative form.\\
Hay otra forma acortada negativa.

But we'll do that in lesson 143.\\
Pero haremos eso en la lección 143.

Excellent! visit my website inglesmuybueno.com to do the interactive
exercise for this lesson and to get the audio file with the examples.

¡Excelente! visita mi sitio web inglesmuybueno.com para hacer el ejercicio
interactivo para esta lección y para conseguir el archivo de audio con los
ejemplos.

There's a link in the description of the video.\\
Hay un enlace en la descripción del video.

In the course so far we've used about 938 words.\\
En el curso hasta ahora hemos usado aproximadamente 938 palabras.
