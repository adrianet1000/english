\section{Lesson 10 ``Preguntas con qué en inglés''}

Hi! How's it going?\\
I'm Andy also known as ``the inglesito'' and I'm here to help you with
your english.

What?\\
¿Qué?

What, tiene el sonido mas parecido a una o del español en el reino unido,
mientras que la pronunciación en estados unidos tiene la pronunciación mas
parecido a una a del español.

What's up?\\
¿Qué pasa?.

What are you doing?.\\
¿Qué estas haciendo?

What is it?.\\
¿Qué es?

¿Qué es esto?\\
What's this?

¿Qué opinas?\\
What do you think?

What's this?\\
¿Qué es esto?

It's a bus.\\
Es un autobus.

What's this?\\
It's a chair.\\
Es una silla.

\textcolor{red}{chair,  se pronuncia como si tubiera dos e del español.
con un sonido largo de la e española, sin llegar a pronunciar la r.}

What's this?\\
It's a school.\\
Es un colegio.

What's this?\\
It's milk.\\
Es leche.

What's this?\\
It's water.

\textcolor{red}{What's termina con el sonido de serpiente.}

What is this?\\
Esta es la forma larga y se usa por razones especiales.

La forma acortada es con mucho la versión mas común en el ingles nativo.

si no tenemos una justificación especial suena raro la version larga.

What is this?

Las contracciones existen para mejorar la fluidez cuando hablamos.

si no queremos hablar de manera fluida no necesitamos una contracción.
Por ejemplo cuando realizamos un juego.

Por ejemplo si lo pregunto dramaticamente jugando un juego.

What isssss thisssss?

y tiene sentido usar una forma larga, no he hablado con fluidez, he hablado
bastante por separado.

hablamos tambien por separado cuando hablamos bastante fuerte, si un padre
le pregunta a su niño.

¡¿Qué es esto?!, con enojo. sentimos mas enojo si se usa una forma larga.

Tambien se suele utilizar la forma larga con los extrangeros para hablar
de manera super clara.

\textcolor{red}{Number, se pronuncia la segunda silaba con el sonido de
cabernicola, es decir que no se pronuncia la r.}

bye!

