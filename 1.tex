\section{Lesson 1 ``Saludar a la gente y decir tu nombre''}

Hi! How are you?.\\
!Hola¡ !que tal¡

I'm Andy and I'm here to help you with your english.\\
Soy Andy y estoy aquí para ayudarte con tu ingles.

Te voy a saludar de dos maneras.

¿Cómo lo hago?.\\
Ready.\\
Listo.

Hi! I'm José.\\
Hello! I'm Pablo.\\
Hi! I'm María.\\
Hello! I'm Roberta.\\
Hi! I'm Paco.\\
Hello! I'm Bibiana.\\

Hi! tiene el sonido como si acabaras de tener una maraton, es decir una
carrera y que estas muy cansado y por lo tanto el sonido de j es como
si te faltara el aire, como una jota suave.

Hello! Nota que en ingles la doble L se pronuncia igual que una L. La letra
doble L no existe en ingles.

En libros antiguos podremos ver que tambien se puede escribir con la letra a
Hallo, pero no es muy comun. Lo normal es verlo escrito con una e, Hello.

Estilisticamente uno de estos saludos es informal y el otro es mas neutro.

Hi! es un saludo informal.

Hello! es un saludo más neutro.

¿Cúal crees que se escucha más?.

Escucharas más ``Hi!'', en el ingles nativo porque es informal.

I'm es con mucho la manera más común para decir ``Yo soy / Yo estoy.'',
es decir la forma acortada de ``I am''.

Di de manera natural ``Soy Alejandro''.

Hi! I'm Alejandro.

Usando un saludo más neutro di ``Hola! soy Alejandro''.\\
Hello! I'm Alejandro.

El apostrofo denota una contracción. El apóstrofo nos indica que algo se
ha quitado y luego las partes se han combinado.

Hello. I am John.

Una de las justificaciones para usar la forma larga de ``hablar super
claramente'', quiza con un extranjero con poco ingles o con alguien que
no escuche bien.

bye.
