\section{Lesson 7 ``Decir cómo te llamas más dos posesivos''}

Hi! How's it going?.\\

I'm Andy also known as ``the inglesito'' and I'm here to help you with your
english.

Soy Andi tambien conocido como ``el inglesito'' y estoy aquí para ayudarte
con tu ingles.

My name's Juan.\\
Me llamo Juan.

Me llamo Melanie.\\
My name's Melanie.

Me llamo Bryan.\\
My name's Bryan.

Me llamo Sally.\\
My name's Sally.

Me llamo Pablo.\\
My name's Pablo.

Me llamo Elena.\\
My name's Elene.

Me llamo Peter.\\
My name's Peter.

Me llamo Viviana.\\
My name's Viviana.

Very good!.\\
¡muy bien!.

This is my brother.\\
His name's John.

Este es mi hermano.\\
Su nombre es John.

This is my dad.\\
His name's Peter.

Este es mi Papa.\\
Su nombre es Peter.

This is my best friend.\\
His name's Pablo.

This is my best friend.\\
her name's Susan.

Esta es mi mejor amiga.\\
Ella se llama Susan.

This is my mum.\\
Her name's Jenny.

Esta es mi mamá.\\
Ella se llama Jenny.

This is my sister.\\
her name's Kylie.

Esta es mi hermana.\\
Ella se llama Kylie.

\textcolor{red}{literalmente en español es traducido como
``su nombre es ...''.}

\textcolor{red}{El inglesito pronuncia la r inglesa solo cuando la r
se encuentra antes de un sonido vocalico.}

His name's Pablo.\\
Her name's Marta.\\

His name's Miguel.\\
El se llama Miguel.

Her name's Juanita.\\

Her name's Ana Maria.\\
His name's Carlos.\\

Hablando sobre el papá de Carlos, di ``su papá''.\\
His dad.\\

Hablando nuevamente sobre el papá de Carlos di de manera natural
``su papá esta aquí.''.\\
His dad's here.\\

Hablando sobre la mamá de Juanita di ``su mamá...''\\
Her mum.\\

En la mayoria del mundo angloparlante se escribe m\textcolor{yellow}{u}m.

Pero en Estados Unidos se dice m\textcolor{yellow}{o}m. con una o.

Hablando nuevamente sobre la mamá de Juanita di de manera natural, es decir
con una forma acortada ``su mamá esta en inglaterra.''.

Her mum's in England.

\textcolor{red}{Es natural contraer is despues de los sustantivos en general
si no impide la fluidez.}

Alguien no te cree. Le repites que la mamá de Juanita esta en inglaterra,
pera esta vez con una forma larga tipo ``si que su mamá esta en inglaterra.''

Her mum is in England.

Hablando sobre el amigo de Miguel, di de manera natural ``su amigo es de
mexico''.

hes friend's from Mexico.

¿Cómo diriamos esto con una forma larga y menos común?

Hes friend \textit{is} from Mexico.

Sobre la mejor amiga de Martha di ``su mejor amiga es de españa.''

Her best friend's from Spain.

Nuevamente la forma acortada es la natural.

Sobre una mujer di ``mi mamá es su mejor amiga.''\\
My mum's her best friend.

De manera natural di ``Eres su mejor amigo.''

Como no hay contexto si es hombre o mujer, no podemos decidir si utilizar
her o his.

Bueno queriendo decir de él di ``Eres su mejor amigo.''.

You're his best friend.

No te cree, con una forma larga di ``si que eres su mejor amigo.''

You \textit{are} his best friend.














